%% Copernicus Publications Manuscript Preparation Template for LaTeX Submissions
%% ---------------------------------
%% This template should be used for copernicus.cls
%% The class file and some style files are bundled in the Copernicus Latex Package, which can be downloaded from the different journal webpages.
%% For further assistance please contact Copernicus Publications at: production@copernicus.org
%% https://publications.copernicus.org/for_authors/manuscript_preparation.html


%% Please use the following documentclass and journal abbreviations for preprints and final revised papers.
% Natural Hazards and Earth System Sciences (nhess)

%% 2-column papers and preprints
\documentclass[nhess, manuscript]{copernicus}


%% \usepackage commands included in the copernicus.cls:
%\usepackage[german, english]{babel}
\usepackage{tabularx}
%\usepackage{cancel}
%\usepackage{multirow}
\usepackage{supertabular}
%\usepackage{algorithmic}
%\usepackage{algorithm}
%\usepackage{amsthm}
%\usepackage{float}
%\usepackage{subfig}
%\usepackage{rotating}

% Set default figure placement to htbp
\makeatletter
\def\fps@figure{htbp}
\makeatother

\begin{document}

\title{A data-driven evaluation of post-fire landslide susceptibility}




%% The [] brackets identify the author with the corresponding affiliation. 1, 2, 3, etc. should be inserted.

% \Author[affil]{given_name}{surname}

\Author[1, 2]{Elsa S.}{Culler}
\Author[1, 2]{Ben}{Livneh}
\Author[1]{Balaji}{Rajagopalan}
\Author[3, 2]{Kristy F.}{Tiampo}


\affil[1]{University of Colorado Boulder Department of Civil,
          Architectural, and Environmental Engineering, USA}
\affil[2]{Cooperative Institute for Research in Environmental Sciences (CIRES), 
          University of Colorado Boulder, USA
}
\affil[3]{University of Colorado Boulder Department of Geologic Sciences, USA}

\correspondence{Elsa Culler (elsa.culler@colorado.edu)}

\runningtitle{A data-driven evaluation of post-fire landslide susceptibility}

\runningauthor{Elsa Culler}


\received{}
\pubdiscuss{} %% only important for two-stage journals
\revised{}
\accepted{}
\published{}

%% These dates will be inserted by Copernicus Publications during the typesetting process.


\firstpage{1}

\maketitle

\begin{abstract}
Wildfires change the hydrologic and geomorphic response of watersheds, which has been associated with cascades of additional hazards and management challenges. Among these post-wildfire events are shallow landslides and debris flows. This study evaluates post-wildfire mass movement trigger characteristics by comparing the precipitation preceding events at both burned and unburned locations. Landslide events are selected from the NASA Global Landslide Catalog (GLC). Since this catalog contains events from multiple regions worldwide, it allows a greater degree of inter-region comparison than many more localized catalogs. Fire and precipitation histories for each site are established using MODIS burned area and CHIRPS precipitation data, respectively. Analysis of normalized seven-day accumulated precipitation for sites across all regions shows that, globally, mass movements at burned sites are preceded by less precipitation than mass movements without antecedent burn events. This supports the hypothesis that fire increases rainfall-driven mass movement hazards. An analysis of the seasonality of mass movements at burned and unburned locations shows that mass movement-triggering storms in burned locations tend to exhibit different seasonality from rainfall-triggered mass movements in areas undisturbed by recent fire, with a variety of seasonal shifts ranging from approximately six months in the Pacific Northwest of North America to one week in the Himalaya region. Overall, this manuscript offers an exploration of regional differences in the characteristics of rainfall-triggered mass movements at burned and unburned sites over a broad spatial scale and encompassing a variety of climates and geographies.
\end{abstract}

\introduction

Mass movements are destructive when they occur near vulnerable areas, causing damage to buildings, utility lines, and roadways \citep{highlandLandslideHandbookGuide2008}. Landslide mitigation costs in the United States (US) are approximately 2 billion USD annually, with worldwide costs much higher \citep{schusterSocioeconomicEnvironmentalImpacts2001}. There can also be indirect impacts, such as aggradation of the streambed, or the formation of landslide dams \citep{glade2005nature}. Worldwide, these natural disasters cause tens of thousands
of deaths each year \citep{froudeGlobalFatalLandslide2018}. Landslide mitigation
costs in the United States (US) are approximately \(2\) billion USD annually, with
worldwide costs much higher \citep{schusterSocioeconomicEnvironmentalImpacts2001}. Though an
accurate assessment of mass movement hazards would aid mitigation efforts
\citep{spikerNationalLandslideHazards2002}, such an evaluation presents a challenge in part because mass movements are often triggered by a sequence of cascading
natural hazards \citep{kloseIntroduction2015}. For example, mass movements may interact with other complex phenomena such as heavy rain, wildfires, floods, earthquakes, melting permafrost
and glacial outbursts \citep{budimirSystematicReviewLandslide2015, harpLandslideInventoriesEssential2011, kirschbaumChangesExtremePrecipitation2020, kirschbaumAdvancesLandslideNowcasting2012, rupertUsingLogisticRegression2003}. 

Here, we focus on a particular sequence of cascading natural hazards known as the post-wildfire landslide. In these events, wildfires are followed by intense precipitation leading to mass movements such as shallow landslides, or debris flows. The impact of wildfires, which themselves occur more frequently and severely as a consequence of higher temperatures and increasingly widespread drought, can lead to a variety of geo-hydrological hazards including increased snowmelt, water contamination, increased erosion rates, and decreased infiltration \citep{aghakouchak2020climate}. Post-wildfire landslides in particular occur when wildfires are followed by intense precipitation, leading to mass movements such as a sediment-laden floods, shallow landslides, or debris flows. We use the term 'mass movement' 
preferentially over landslide in this study in order the encompass this variety 
of phenomena, not all of which are landslides in the strictest definition. Nonetheless, 
when describing prior literature in which 'landslides' and 'mass movements' are used interchangeably we defer to the terminology used by the authors of the cited work. 

The impact of wildfire on landslide hazards can vary on the basis of static factors 
such as burn severity, vegetation, and soil types \citep{cannonPredictingProbabilityVolume2010,staley2018estimating}. Mass movement hazards in general may also depend on dynamic factors such as soil moisture, meteorology and
the length of time since the most recent fire \citep{kirschbaumSatelliteBasedAssessmentRainfallTriggered2018,degraffTimingSusceptibilityPostFire2015,mcguire2021time}. There are numerous local studies demonstrating a relationship between wildfire occurrence or severity and the amount of precipitation that triggers a mass movement \citep{cannonStormRainfallConditions2008,gartnerRelationsWildfireRelated2005, reneauSedimentDeliveryWildfire2007,riley2013frequency} . The impact of wildfire on landslide hazards can also vary on the basis of local factors such as vegetation, and soil type \citep{cannonPredictingProbabilityVolume2010, staley2018estimating}. In general, the lack of complete landslide inventories including a wide variety of climates and ecoregions presents an obstacle to evaluating the role of fire in rainfall-triggered landslides\citep{kloseMethodology2015}.

This study seeks to test the hypothesis that wildfire consistently increases mass movement susceptibility across six global regions by detecting and characterizing differences in mass movement-triggering precipitation at both burned and unburned sites. Though we cannot draw conclusions about the susceptibility leading to any particular event, less precipitation among a group of post-wildfire suggests that the threshold for triggering a mass movement was lowered, e.g. susceptibility was greater. A second purpose of this study is to explore the possibility that the relationship between wildfire history and the expected frequency of landslide-triggering precipitation varies by region.

The GLC provides a large collection of rainfall-triggered landslides taking place in a variety of climates such that, in combination with spatially continuous observations of fire (500m Moderate Resolution Imaging Spectroradiometer [MODIS] Burned Area by \citet{giglioCollectionMODISBurned2018}) and precipitation (5.5km Climate Hazards group InfraRed Precipitation with Station data [CHIRPS] by \citet{funkClimateHazardsInfrared2015}) data, it is well suited for comparing the diverse precursors under which post-wildfire mass movements occur.




\subsection{Mechanisms by which fire increases mass movement hazards}
\label{mechanisms-by-which-fire-increases-landslide-hazards}

While many factors contribute to mass movement hazards, only a subset are altered by fire
exposure \citep{highlandLandslideHandbookGuide2008}, and are therefore of interest to this
analysis. Fire changes hydrologic and geomorphic response through several distinct physical mechanisms. First, the
destruction of vegetation contributes to the development of debris flows
and other mass movements in three ways:

\begin{itemize}
\item
  Sediment gathered behind vegetation trunks and stems can, after 
  a fire, be mobilized  either by a rain storm or as dry ravel, i.e. sediment
  that rolls down the slope without precipitation \citep{cannonWildfirerelatedDebrisFlow2005}.
\item
  Vegetation destruction clears pathways for water and sediment to flow
  downhill more quickly \citep{shakesbyWildfireHydrologicalGeomorphological2006}.
\item
  Following a fire, canopy and litter storage - water
  that gets trapped in leaves and other detritus on the ground - is
  greatly reduced, resulting in increased runoff and sediment
  transport \citep{cannonWildfirerelatedDebrisFlow2005, shakesbyWildfireHydrologicalGeomorphological2006}.
\end{itemize}

Additionally, soil properties can be dramatically altered post-fire, resulting in
the following changes which can affect the formation of
mass movements:

\begin{itemize}
\item
    Burned soils can have reduced organic content as a result of the
    combustion process, which causes them to have reduced water-holding
    capacity \citep{nearyWildlandFireEcosystems2005}.
\item
    Combustion of organic content also typically reduces soil aggregate
    stability, promoting erosion \citep{shakesbyWildfireHydrologicalGeomorphological2006}.
\item
    Some combinations of soil, vegetation type, and temperature can
    decrease wettability or produce a hydrophobic layer 1-5 cm beneath the
    soil, thereby dramatically increasing runoff \citep{spittlerFireDebrisFlow1995}. The
    implications of this effect vary from place to place,
    since fire can also destroy hydrophobic layers in the right conditions. 
    In addition, these effects are not always
    uniform across the burned area, and the effects of changed wettability
    can last from days to years depending on the local conditions
    \citep{shakesbyWildfireHydrologicalGeomorphological2006}.
\item
    A layer of post-fire ash caused by fire can also increase soil storage potential depending 
    upon the thickness and hydraulic conductivity of the layer \citep{ebelHydrologicConditionsControlling2012}.
\end{itemize}

One consequence of wildfire-driven changes to soil and vegetation on rainfall-triggered mass movements is that the predominant mechanism shifts from infiltration-driven to runoff-driven \citep{cannonWildfirerelatedDebrisFlow2005}. Infiltration-driven mass movements are typically shallow slope failures initiated by longer storms that saturate the shallow subsurface. By contrast, runoff-driven mass movements are often debris flows caused by high-volume storms that mobilize sediment on the surface without the need for much infiltration. Mass movements can often be identified as one type or another primarily by observing whether it had a point origin, as with infiltration-driven mass movements, or a distributed origin like runoff-driven mass movements. For infiltration-driven mass movements, the antecedent soil moisture conditions are more important for evaluating mass movement hazards since soil saturation is fundamental to the mechanism of slope failure. However, post-wildfire mass movements tend to be less driven by infiltration since the hydrophobic and more erodible sediment layer creates an ideal condition for runoff-driven mass movements. \citep{cannonStormRainfallConditions2008,santi2020wildfire,parise2012wildfire}.

\subsection{Evidence for increased mass movement hazards with increased burn
severity}\label{evidence-for-increased-mass movement-hazards-with-increased-burn-severity}

Wildfire has been empirically linked to increased frequency and volume
of debris flows in several regions of the Western US \citep{cannonWildfirerelatedDebrisFlow2005}. A key piece of evidence for this connection comes from a
series of studies based on repeated post-storm observations of burned
watersheds in Southern California and the Intermountain West regions of the US as part of
the development of the US Geological Survey's (USGS)
operational post-wildfire mass movement hazard predictions \citep{cannonPredictingProbabilityVolume2010, gartnerMutivariateStatisticalModels2009, gartnerEmpiricalModelsPredicting2014, rupertUsingLogisticRegression2003, staleyUpdatedLogisticRegression2016}. These five studies model the probability of mass movements following
fire using logistic regressions to demonstrate that both burn severity
\citep{staleyUpdatedLogisticRegression2016} and burn extent within a watershed \citep{cannonPredictingProbabilityVolume2010} are associated with increased debris flow likelihood. Notably, burn severity and extent are both increased by drought and other low antecedent soil moisture \citep{westerling2003interannual}, and thus we expect to find more post-wildfire debris flows in dry climates. \citet{gartnerEmpiricalModelsPredicting2014} found that the increase in debris flow probability in a
watershed due to wildfire is greatest immediately after wildfire, but can last a total of 2-5 years. Other studies suggest that the overall mass movement hazard evolves over time in a more complex manner, with debris flow hazards increasing for the year after the fire followed by an increase in the frequency of shallow landslides as tree roots decay in subsequent years \citep{rengersLandslidesWildfireInitiation2020,benda1997stochastic}. Increased likelihood of post-wildfire debris flows has also been associated with the erodibility of fine sediment in the soil, soil organic matter percentage, soil clay percentage, underlying lithology (e.g. sedimentary or granitic rock), watershed area, and watershed relief ratio \citep{gartnerMutivariateStatisticalModels2009, rupertUsingLogisticRegression2003,pelletier2014sediment}.

The widely recognized relationship between mass movements
and burn severity suggests that mass movement susceptibility
increases after wildfires in the Western US, although none of the
above studies include observations of unburned sites as a control.
Instead, the databases used in \citet{cannonPredictingProbabilityVolume2010, gartnerMutivariateStatisticalModels2009, gartnerEmpiricalModelsPredicting2014, rupertUsingLogisticRegression2003, staleyUpdatedLogisticRegression2016} include multiple
observations of the presence or absence of a debris flow at each site,
making them suitable for a regression analysis based on burn severity,
but not for comparing burned and unburned locations. In addition, while
these post-wildfire mass movement observations contain precise dates and locations, and extend
across a remarkable spatial range when compared to most other mass movement hazard
models, they still are limited to 119 sites or fewer \citep{gartnerEmpiricalModelsPredicting2014}. This limited spatial extent leaves open the question of whether
the fire-flood patterns of the Western US are unique, or if similar
hazards are just as ubiquitous but under-reported in other regions. A global study by \citet{riley2013frequency} comparing post-wildfire a non-fire-related 
debris flows found that the volumes of the post-wildfire debris flows tended to be smaller.
This finding suggests that the increase in debris flow hazard and frequency after wildfires occurs in a variety of environments.

\subsection{Sources and methods for mass movement data
collection}\label{sources-and-methods-for-mass-movement-data-collection}

It is resource-prohibitive to conduct a continuous systematic search for mass movements either in the field or with satellite observations. As a result, many of the most accurate and complete methods for systematically identifying mass movements can presently only be used over limited spatial and temporal domains. For example, \citet{leeLandslideHazardMapping2007} identified landslides from aerial photograph interpretation and a field survey over the \(\sim800 \text{ km}^2\)
Selangor area in Malaysia, and \citet{nefesliogluAssessmentLandslideSusceptibility2010}, used an inventory based on aerial photographs taken in 1955-1956 to analyze landslide susceptibility over a \(\sim175 \text{ km}^2\) area near Istanbul, Turkey. An alternative to manual identification either in the field or using photographs is automatic or semi-automatic landslide detection using image processing on aerial imagery, LiDAR surveys, or Synthetic Aperture Radar (SAR). These automated methods are typically applied over similarly small domains due to challenges with obtaining imagery and compiling training datasets. For example, \citet{marthaLandslideHazardRisk2013} used aerial imagery over \(\sim120 \text{ km}^2\) in the Himalayas, while \citet{mezaalOptimizedNeuralArchitecture2017} used LiDAR over the \(26.7 \text{ km}^2\) Cameron Highlands of Malaysia. SAR interferometry can be used for identification of pre-landslide motion, as was done by \citet{luPersistentScatterersInterferometry2012} over the \(\sim1500 \text{ km}^2\) Arno basin in Italy. In addition, several SAR techniques have been employed to identify post-landslide scars, including SAR amplitude mapping of landslides triggered by the Gorkha, Nepal earthquake in \(2015\) over a \(14,500 \text{ km}^2\)
area \citep{meenaComparisonEarthquakeTriggeredLandslide2019}, coherence mapping of interferometric SAR the same earthquake-triggered landslides \citep{burrowsNewMethodLargeScale2019}, and the wildfire-triggered landslides over \(\sim60 \text{ km}^2\)
of the area burned by the \(2017\) Thomas Fire in California \citep{donnellanUAVSAROpticalAnalysis2018}. While automated mass movement detection as deployed in the above studies is continually undergoing promising advances, at the time of this analysis it has not yet been used to compile an inventory over a broad enough spatial domain to facilitate inter-regional comparisons. Such records collected in an uncoordinated effort over small domains are unsuitable for regional inter-comparisons such as we have undertaken here because these records do not contain standardized information for every region, are often unpublished \citep{vanwestenLandslideHazardRisk2006}, and are unlikely to have daily temporal resolution that would allow comparison with the precipitation record \citep{kirschbaumGlobalLandslideCatalog2010}.

For this study, we chose to use the NASA Global Landslide Catelog
\citep[GLC,][]{kirschbaumGlobalLandslideCatalog2010}. As with the few 
other regional and global databases available, the broad domain of the 
GLC comes coupled with issues of location error and spatial bias. For 
each landslide location, the GLC contains an estimate of the area in 
which the landslide occurred, labeled the “location accuracy”. For 
consistency, we refer to this parameter using the same name. The sources of GLC data are
second-hand observations made by the news media, governmental
organizations such as departments of transportation, and some available
scientific reports \citep{kirschbaumGlobalLandslideCatalog2010}. The absence of a
systematic search for mass movements across the entire database domain
results in a substantial spatial bias towards populated areas where
mass movements happen to be noticeable. News reports also suffer relatively
high location uncertainty (as much as 50 km) depending on how specific the
source article is about the location \citep{kirschbaumGlobalLandslideCatalog2010}. Finally,
though the GLC does contain some information about the mass movement
mechanisms that would allow mass movements to be classified, for example, as
debris flows or shallow landslides, the majority of the events in the GLC are labeled as a non-specific 'landslide' type, which could refer to any type of mass movement.  Despite limitations in accuracy
and completeness, the GLC was
chosen for this study primarily because as of this writing it offers the largest spatial and
temporal range of any catalog. The GLC contains a
sample of mass movements from across the globe
\((n=11377, 5313 of which met study requirements - see Section \ref{sec:landslide-data})\), and a substantial proportion of mass movements were identified in
this study as having occurred in recently burned areas (\(n=489\); \(9.2\,\%\)).


\subsection{Towards a global picture of mass movement
susceptibility}\label{toward-a-global-picture-of-global-landslides-susceptibility}

This study seeks to test the hypothesis that wildfires increase
landslide susceptibility by evaluating antecedent precipitation at both
burned and unburned mass movement locations. Some existing
local and regional studies \citep{cannonPredictingProbabilityVolume2010, rupertUsingLogisticRegression2003} have assessed the impact of wildfire on mass movement susceptibility, but have not included unburned locations in their analyses. Other
studies have also featured the GLC data and a global spatial extent,
with a focus on validating large-scale mass movement hazard models
\citep{kirschbaumSatelliteBasedAssessmentRainfallTriggered2018}. This analysis is unique from other
regional and global studies in that it combines the broad scope of the
GLC data with an exploration of the role of wildfire in mass movement
susceptibility. This study is also distinct from others that focus on
the role of wildfire on mass movement sites \citep{gartnerMutivariateStatisticalModels2009} in
that here burned sites are contrasted with unburned sites instead of
previous observations of the same location. Finally, in contrast to
post-wildfire mass movement studies focused on a specific regions like the
western US \citep{cannonIncreasingWildfirePostFire2009}, southern California 
\citep{gartnerEmpiricalModelsPredicting2014}, Western Canada \citep{jordan2015post}, 
Korea \citep{lee2019analysis} or southeast Australia 
\citep{nymanEvidenceDebrisFlow2011}, this study
combines the GLC with globally-observed fire and precipitation data to offer
unique insights into the role of fire on mass movement susceptibility in
diverse regions across the globe.

\section{Methods}\label{methods}

We first describe the mass movement data (Sect. \ref{sec:landslide-data}), the study regions 
(Sect. \ref{sec:study-regions}) and 
fire data (Sect. \ref{sec:fire-data}). Mass movements were included only if precipitation data and at least 3 years antecedent fire data were available. The mass movements occurred between 2007 and 2019, with corresponding precipitation and fire data extending as far as 2004-2019 so as to capture antecedent conditions. The precipitation data 
(Sect. \ref{sec:precipitation-data}) leading up to the date of each mass movement were 
compared using three approaches.  First, the seven-day running total precipitation depth percentile for the 30 days surrounding the day of the year and across the total 38-year record  (see Sect. \ref{sec:precipitation-data})
was used as a proxy for mass movement
susceptibility. We assume here that greater susceptibility results in a lower precipitation threshold to trigger a landslide. An observation, therefore, of lower precipitation percentile values triggering mass movements across a sample of sites suggests that susceptibility is generally higher in that group. This principle is illustrated in the susceptibility-based rainfall threshold model developed by \citep{monsieurs2019susceptibility}, in which the predicted threshold of antecedent rainfall resulting in a landslide is adjusted according to susceptibility factors. This percentile value was compared between burned and
unburned sites within each region and for all included mass movements (see Sect. \ref{sec:method-percentile}). Next,
seven-day precipitation percentiles were compared with bootstrapped
samples from burned and unburned sites separately (see Sect. \ref{sec:method-bootstrap}) to
confirm the findings from the depth percentile analysis and also to draw out
differences in storm timing between burned and unburned groups. Finally,
the precipitation frequency in the burned and unburned groups in the
months and years surrounding each mass movement (see Sect. \ref{sec:method-seasonality}) was examined
to identify shifts in the seasonality of mass movements at burned sites
relative to the unburned group. These seasonality results were augmented
with kernel density estimates of mass movement occurrence by day-of-year at
burned and unburned sites for each region.

\subsection{Mass movement data}\label{sec:landslide-data}

\begin{figure}
\centering
\includegraphics[width=0.8\textwidth]{./figures/panel_map.png}
\caption{Landslide locations (\(n=5313\), \(2006-2017\)),
showing region coding (see Sect. \ref{sec:study-regions}) in (a) and (b), with 
location accuracy for burned and unburned groups in the regional insets; 
burned/unburned classification at the time of the mass movement in (c) and (d),
with regional insets showing kernel density portrayal of the fraction 
of burned area for the mass movement locations from the three years 
preceding the mass movement; and the precipitation 
percentile on the day of the mass movement in (e) and (f), with regional insets 
of kernel density estimates (violin plots) of the climatological (\(1981-2020\)) 
seasonal precipitation magnitude (mm) including a reference line indicating the 
median seasonal average across all sites globally. Country boundaries were obtained from the \texttt{maps} R package \citep{deckmynMapsDrawGeographical2018}}
\label{fig:map}
\end{figure}

A sample (\(n=5313\)) of rainfall-triggered mass movements was obtained
from the GLC. Mass movement locations are shown in Fig. \ref{fig:map},
along with a summary of fire and precipitation information obtained for
those locations from the sources listed in Table \ref{tbl:datasets} (see
Sects. \ref{sec:fire-data} and \ref{sec:precipitation-data}). The GLC provides a large collection
of events taking place in a variety of climates such that, in
combination with spatially continuous observations of fire 
\citep[\(500 \text{m}\) Moderate Resolution Imaging Spectroradiometer {[}MODIS{]} Burned Area by][]{giglioCollectionMODISBurned2018} and precipitation 
\citep[5.5km Climate Hazards group InfraRed Precipitation with Station data {[}CHIRPS{]} by][]{funkClimateHazardsInfrared2015}
data, it is well suited for comparing the diverse precursors under which
post-wildfire mass movements occur.

In order to reduce errors resulting from including a variety of types of rainfall-triggered mass movements within the same dataset, the selected mass movements were limited to  those labeled in the GLC with a `landslide
trigger' value of `rain,' `downpour,' `flooding,' or `continuous rain.' Mass movements with a second trigger such as an earthquake were eliminated.
Snowmelt-driven mass movements were also not included because the impact of
precipitation can be delayed in those cases. An analysis of the snow record
in California/Nevada revealed only a single event with enough antecedent
snow to suggest it could have been mislabeled. Only records with location 
uncertainty of \(10 \text{ km}\) or less 
were included, since the mass movements with lower location accuracy presented
problems for wildfire classification. Finally, only mass movements
between \(50^{\circ}\text{S}\) and \(50^{\circ}\text{N}\) latitude were included, and the
events occurring before the year 2000 were omitted, so as to ensure
coverage by both fire and precipitation datasets (see
Table \ref{tbl:datasets}).

The GLC contains a variety of types of rainfall-triggered mass movements with different physical mechanisms, including debris flows, shallow landslides, and rock falls. The majority of included mass movements (65 \%), however, are categorized simply as ‘landslide’, which according to the dataset authors can mean any type of mass movement. Since most of the mass movements are of an unknown type, we did not exclude data on the basis of category. Of the specific types of mass movements, most are labeled mudslides (25 \%), with the next largest category being rockfalls at 4 \%. This uncertainty as to landslide mechanism is currently a necessary trade-off for large spatial scales. This limitation highlights the need for large-scale catalogs for specific types of mass movements, such as debris flows or shallow landslides.

\begin{table}
\centering
\caption{Description of datasets used in the analysis}
\label{tbl:datasets}

\begin{tabularx}{\textwidth}{
    |>{\hsize=1\hsize\linewidth=\hsize\raggedright\arraybackslash}X
    |>{\hsize=1.75\hsize\linewidth=\hsize\raggedright\arraybackslash}X
    |>{\hsize=.75\hsize\linewidth=\hsize\raggedright\arraybackslash}X
    |>{\hsize=1\hsize\linewidth=\hsize\raggedright\arraybackslash}X
    |>{\hsize=.75\hsize\linewidth=\hsize\raggedright\arraybackslash}X
    |>{\hsize=.75\hsize\linewidth=\hsize\raggedright\arraybackslash}X|}

\tophline
Data source &
Description &
Spatial extent &
Spatial Resolution &
Temporal Range &
Temporal Resolution \\ \hline\hline

NASA Global Landslide Catalog (GLC; Kirschbaum et al., 2010) &
Compilation of landslides drawn from news articles and scientific
reports &
Global, with variable coverage in different countries &
Landslide location accuracy varies from exact to 50 km range. The
coarsest location accuracy used was \(10\text{ km}\). &
1988--2015, most data 2007--2015 &
Daily for most data points \\\hline

Climate Hazards Infrared Precipitation with Stations (CHIRPS) (Funk et
al., 2015) &
Station-corrected gridded precipitation data derived from cloud
temperature observed using infrared satellite observations &
\(50^{\circ} \text{S}\) to \(50^{\circ} \text{N}\) &
\(0.05^{\circ}\) (\(\sim5.5\text{ km}\)) &
1981--2020 &
Daily \\\hline

MODIS Burned Area (Giglio et al., 2018) &
Dates on which a pixel was burned, derived from NASA’s MODIS Terra and 
Aqua satellites. The product uses a reprocessing algorithm that combines
changes in burn-sensitive vegetation index and active fire locations.  &
global &
\(500 \text{ m}\) &
2000--2020 &
Daily \\\hline

Daymet Precipitation and Snow Water Equivalent (Thornton et al.,
2014) &
An alternative precipitation dataset based on station data and
topographic information &
North America &
\(1 \text{ km}\) &
1980--2020 &
Daily  \\\hline
\end{tabularx}
\end{table}


\subsection{Study regions}\label{sec:study-regions}

To compare the differences in mass movement triggers in different climates,
we divided the mass movements into regions (see Fig. \ref{fig:map}
panels (a)and (b)). Regions were determined using the AGglomerative NESting
(AGNES) hierarchical clustering algorithm \citep{kaufman2009finding}
considering the latitude and longitude of the mass movements, and clusters were subsequently
combined, split, or eliminated on the basis of equalizing sample sizes as described below. 
Though the regions are still large enough to encompass considerable variability in 
climate, the spatial clustering helps to ensure that the variability across 
regions - particularly in latitude - is larger than the variability within.

First, the cluster tree was
truncated at 30 clusters, after which all the clusters with fewer than
100 data points or less than \(5\,\%\) burned sites were eliminated. Notably, two commonly studied regions for mass movements - Europe and Australia \citep[e.g. ][]{vandeneeckhautStateArtNational2012,nymanEvidenceDebrisFlow2011} – were eliminated at this stage due to a lack of verifiable post-wildfire mass movements available in the GLC. Cases where
two nearby regions both had lower numbers of mass movements, for example, Central America and
Caribbean/Venezuela, were joined manually. Finally, the largest region, encompassing 
Western US and Canada, was split into three sub-regions based on an additional identical clustering 
process over this sub-domain. The final regions are shown in
Fig. \ref{fig:map} panel (a). The Pacific Northwest of North America was included
even though the percentage of burned sites is lower than threshold,
but at \(4.4\,\%\) it was nearly double the highest percentage
among the eliminated regions (\(2.25\,\%\) in the Eastern US). Some mass movements
were not included in any of the final regions. These events were not, however,
eliminated from any analysis of all mass movements.

\subsection{Fire data}\label{sec:fire-data}

For each mass movement, a circle centered at the mass movement location and with
a radius of the location accuracy was computed and each $500\,\text{m}$ pixel 
within the circle was extracted from the MODIS Burned Area dataset
\citep{giglioCollectionMODISBurned2018}. Fire affects the landscape over a large
range of temporal scales in different settings. Previous studies suggest
that the post-wildfire increase in mass movement susceptibility peaks within the 
first six months, but that a second time period of increased susceptibility 
can appear at 3 years or even longer as a result of root decay
\citep{degraffTimingSusceptibilityPostFire2015, gartnerEmpiricalModelsPredicting2014}. 
Landslides were classified as burned if any part of
the area where the mass movement occurred was burned at some point 
within the three years prior to
the event to capture both waves of increased susceptibility without 
over-identifying mass movements areas where fires occur every few years. 
The fraction of pixels that were burned over the 3-year
antecedent period was then computed, and mass movements classified as burned
if there was any overlap between burned areas and the mass movement circle.
As a result of this analysis, 489 mass movements (\(9.2\,\%\)) were categorized
as potential post-wildfire events.

While this method of identifying post-wildfire mass movements ensured that
all post-wildfire mass movements were classified as burned, the low spatial
accuracy of many of the mass movement locations leaves open the possibility
that some mass movements occurred near a recent fire but not within the fire
perimeter. Due to uncertainty in the exact location of many of the mass movement 
locations, both false positive and false negative errors in burn history 
classification are possible. Some mass movements classified as burned may 
have occurred near a recent fire but not within the fire perimeter, or 
conversely some mass movements classified as unburned may in fact have 
been located inside a fire perimeter but near the edge. However, by 
classifying mass movements as burned if any part of the potential location 
was burned limits the potential for false negative errors while increasing 
the possibility of false positive errors. For this reason we refer to mass 
movements as ‘burned’ instead of post-wildfire in this analysis. Also 
important to note is that false positive burned classification is a 
function of both the burned fraction and the conditional probability 
of mass movement occurrence given that a fire has occurred. False positives 
are therefore
less likely for mass movements with better location accuracy, which made up a 
larger proportion of mass movements in the regions within the US and Canada than other regions.
Fig. \ref{fig:map} shows the distributions of burned fractions for
each region. Note that in Central America and Southeast Asia, very few
sites have above 10 \% burned fraction (see Fig. \ref{fig:map}
panels (c) and (d) inset plots). This could be due to those regions having 
lower mass movement location accuracy, resulting in a higher likelihood of 
false positive post-wildfire mass movements. 

To explore the effects of variability in location accuracy and mass movement type within the GLC, validation analyses were performed to quantify the extent of errors due to these factors. Firstly, the percentages of burned sites in each region were computed for each location accuracy. Subsequently, the results of the Mann-Whitney hypothesis tests comparing pre-landslide precipitation percentiles were duplicated splitting the data in the high- and low-accuracy groups ($<=1\textrm{ km}$ and $> 1\textrm{ km}$ respectively). The number of days with statistically significant differences in precipitation percentile in the 14 days prior to the mass movement and 7 days are computed in each group. Finally, a similar analysis compared debris flows (labeled as ‘debris flow’ or ‘mudslide’ in the GLC) and other types of mass movements.


\subsection{Precipitation data}\label{sec:precipitation-data}

Time series of precipitation at the mass movement sites were obtained from the 
CHIRPS precipitation dataset \citep{funkClimateHazardsInfrared2015}. CHIRPS is a gauge-corrected global precipitation database derived from satellite-based cloud temperature measurements. The CHIRPS dataset was chosen because of its global coverage and relatively long climatological record (1981-present). Though the $\sim5.5\textrm{ km}$ resolution of CHIRPS may present challenges in capturing high-intensity storms that sometimes trigger landslides \citep{hongUseSatelliteRemote2007}, \citet{gupta2020assessment} found that CHIRPS performed well in detecting extreme precipitation across India. Furthermore, this resolution matches the 5 km resolution of the plurality of records in the GLC. 
Precipitation was averaged for each mass movement location within the radius
of the provided location accuracy. Additional pre-processing steps described 
below were performed to distinguish anomalously high precipitation events from 
potential seasonal shifts and climatic differences across sites. 

Mass movements can be triggered by intense and short storms, by long storms of 
lower intensity, or somewhere in-between. A 7-day running average of antecedent 
precipitation was computed to enable direct comparison of the mass movements 
triggered by storms a range of durations. While including an estimate of the 
soil moisture was outside the 
scope of this study, 7-day antecedent rainfall indices consisting of a weighted 
average of precipitation over the 7-day time period have been used by other 
modelling studies as a surrogate for soil moisture in a combined indicator of 
landslide susceptibility  
\citep{jamesAntecedentMoistureConditions2009, kirschbaumSatelliteBasedAssessmentRainfallTriggered2018}. 
Furthermore, 7-day sums of precipitation have been found to perform better 
than other durations in threshold models of landslide occurrence 
\citep{krkavc2017method, garcia2015rainfall}.
Figure \ref{fig:map} panels (e) and (f) show these 7-day cumulative-precipitation 
percentiles, as well as the climatological seasonal average precipitation, revealing 
that the Western US is dominated by dry summers, while the lower-latitude regions 
exhibit wetter summers and in some cases monsoons.

Upon computing the CHIRPS precipitation measurements for each event, we noted 
that some of the categorized rainfall-triggered mass movements in fact had no recorded antecedent 
precipitation in the 7-day window. We screened these such mass movements from the analysis. 
Figure 2 shows a quality control 
sub-analysis for the California/Nevada area to investigate the need for data 
screening on the basis of inconsistencies between the reports of rainfall-triggered 
mass movements and the precipitation 
record. This region was chosen for the quality control analysis, because of its 
high density of precipitation data and variety of climate conditions, useful for 
identifying erroneous mass movement precipitation. We found \(14\,\%\) (\(73\) of \(533\)) 
of the mass movements in this region had no triggering precipitation event recorded in 
the CHIRPS data. Since the GLC contains only rainfall-triggered mass movements, the lack 
of precipitation in these cases was likely a result of errors in either the precipitation 
data or mass movement data. 

A comparison with the Daymet precipitation dataset over the same 
domain revealed that the two precipitation datasets frequently did not agree on these 
zero-precipitation mass movement events, suggesting that the problem largely originated from 
the precipitation data themselves. Daymet is higher-resolution than CHIRPS (1 km vs. 5.5 km) and is based on precipitation gauge measurements. The extent of Daymet is limited to North America and thus is only used for validation in the California area. Furthermore, the concentration of data points on the 
x and y axes of Fig. \ref{fig:chirps-daymet} suggests that disagreements on precipitation 
occurrence are distinct from disagreements on the non-zero amounts of precipitation and 
potentially a separate source of error. To limit the effect of these inconsistent data 
points on the results, all mass movements worldwide with no measured precipitation in the 
six days before and one day after the event were removed from the global study (\(367\) 
of \(5680\) or \(6.5\,\%\) removed for a final \(n=5313\)). 

\begin{figure}
\centering
\includegraphics[width=0.5\textwidth]{figures/chirps_vs_daymet.png}
\caption{Seven-day precipitation percentiles for Daymet versus CHIRPS products computed 
for the six days before and one day following recorded California/Nevada mass movements. 
Blue and black points show the screened and included mass movements, respectively, whereas 
cumulative precipitation from the rest of the available record is shown in grey.}
\label{fig:chirps-daymet}
\end{figure}

Precipitation data were further processed to facilitate the comparison of mass 
movement-triggering events across a variety of seasons and climates. In an initial 
analysis of the precipitation data, we were unable to distinguish between normal 
seasonal increases in precipitation and specific mass movement-triggering precipitation. 
In order to isolate triggering storms, it was necessary to normalize for both 
location and time of year. We accomplished this by computing a 30-day rolling 
percentile of the 7-day running precipitation values based on 38 years of historical precipitation climatology from 1981–2019 for each location. Percentiles have been used to compare landslide-triggering precipitation across larger, e.g. country-sized regions \citep{kirschbaumChangesExtremePrecipitation2020, araujo2022impact} in order to control for differences in climate or precipitation data source. For this study, the percentile produced a uniform distribution of precipitation ranging from 0 to 1, controlling for geographic and seasonal differences. As a result, anomalous precipitation events are highlighted, facilitating the comparison of mass movement triggers across locations and seasons.


\subsection{Precipitation percentile experiment} \label{sec:method-percentile}

This experiment compares the 7-day precipitation percentile in the burned and unburned 
groups in the time leading up to a mass movement. The percentile indicates the degree to 
which mass movement-triggering storms were exceptionally large and also serves as a proxy 
for relative mass movement susceptibility. A one-sided Mann--Whitney hypothesis test was 
used to ascertain whether the precipitation percentiles of burned sites were less than 
the precipitation percentiles of unburned sites. Deviations between the burned and 
unburned groups defined by a p--value less than \(0.05\) on the Mann--Whitney test 
indicate statistically significant differences in the mass movement susceptibility of the 
two groups. The null hypothesis of the Mann–Whitney test was that the distribution 
of precipitation percentile of the burned sites is generally greater than or equal 
to the distribution of precipitation percentiles of the unburned sites \citep{helsel2020Statistical}. 
Percentiles are by definition uniformly distributed rather than normally distributed, 
making the Mann–Whitney test, since it does not require normal distribution, the most 
appropriate hypothesis test for these data. However, since zero-precipitation 
periods are excluded, this method cannot account for differences in the frequency 
of precipitation across different climates, but rather reflects differences in the magnitude 
of 7-day precipitation totals.


\subsection{Bootstrapped samples
experiment}\label{sec:method-bootstrap}

In order to evaluate how anomalous the precipitation events  preceding burned and unburned 
landslides were to “typical” local climate conditions at the mass movement locations, we compared 
them to bootstrapped samples from other years to obtain a clearer signal.  One
hundred samples were taken from the 38-year precipitation records to
match the locations and DOY of the observed mass movements, but
from randomly selected years (\(n=\)the smallest number above 100 that ensured each site was included in the same number of samples).
Sampling was repeated for burned and unburned groups within each region
as well as for all the mass movements in the study. These samples are
representative of precipitation for a particular number of days before the mass movement
and serve as a control dataset with which to compare
the pre-landslide precipitation. Next, the observed event-year
precipitation across all sites in the group was tested against each
bootstrap sample using a Mann--Whitney test, with the null
hypothesis that the sample median precipitation percentile was less than or equal to the median
of the precipitation percentiles from that day of the year in the entire
record from 1981--2020. This produced a distribution
of p--values that represent the likelihood that the precipitation leading
up to the mass movements varied from the control baseline.

This sampling method, though more complex, helps to reduce noise in the
hypothesis test results due to different sample sizes in different
regions. It also provides more information on general mass movement
susceptibility of each region rather than only the relative
susceptibility of burned and unburned sites. Finally, it includes
measurements of zero precipitation, which were eliminated from
the direct comparison because of long-term climatic differences in
precipitation frequency between burned and unburned sites in all
regions.


\subsection{Mass movement seasonality experiment}\label{sec:method-seasonality}

The probability of landslide occurrence in a given temporospatial domain varies 
throughout the year \citep{stanley2020building}; we refer to this annual pattern 
for a given domain as mass movement seasonality. We hypothesize that wildfire 
alters mass movement seasonality. To test this hypothesis, we estimated precipitation 
frequency at the mass movement sites over time by computing
the fraction of sites in the burned and unburned groups that had
precipitation on any given day. As with the percentiles and the bootstrap
p--values, frequency estimates were computed relative to the mass movement
event rather than by calendar date, resulting in time coordinates
measured in `years before the event'. Precipitation frequency was estimated for two years 
before and after the mass movement in order to highlight changes in the magnitude 
and phase of the precipitation pattern. We found that
in most regions there was a long-term difference in the mean annual
precipitation frequency, likely because fires occur more often in areas with drier
climates \citep{liuWildlandFireEmissions2014} and drought 
\citep{balling1992climate,gudmundsson2014predicting}. These persistent differences 
between burned and unburned sites were removed by standardizing the mean 
precipitation frequency for both the burned and unburned groups, that is 
to say subtracting the mean and frequency and dividing by the standard deviation. 
Finally, we took a
90-day running average to reduce noise in the data and thereby make it
easier to visually identify any long-term shifts in mass movement
occurrence. These frequency estimates are not normalized by season,
which means that unlike the previous two metrics they can be used to
compare the degree of shift in the seasonality of mass movements at burned
versus unburned sites relative to annual precipitation cycles.

Additional seasonality analysis was performed to provide insight into the times
of year that mass movements occur at burned versus unburned sites. Kernel density 
estimates of mass movement occurrence throughout the year were compared between the 
burned and unburned groups. This seasonality analysis would highlight a shift 
from Fall to Spring but, in contrast with the frequency analysis, it does not 
indicate the precipitation conditions under which mass movements typically occur. 
Together, the frequency and seasonality analyses can show both the seasonal 
shift as well as any changes in mass movement occurrence relative to annual 
precipitation patterns. 


\section{Results}\label{results}

\subsection{Precipitation percentile experiment}\label{sec:result-percentile}

\begin{figure}
\centering
\includegraphics[width=\textwidth]{figures/magnitude_regional.png}
\caption{Seven-day precipitation percentile in the lead-up to mass movements for 
all mass movements in (a) and for the six individual regions labeled (b)--(g), whether 
classified as part of one of the regions or not. The day of the mass movement is indicated with a vertical grey column. Days where a significant difference 
was found between the burned and unburned groups are indicated in darker colors
(Mann--Whitney hypothesis test, \(p > 0.05\)).}
\label{fig:percentile}
\end{figure}

The distributions of precipitation event percentiles for all the
included mass movements are shown in Fig. \ref{fig:percentile}. The
precipitation percentile increases for all groups as the date of the
landslide approaches, confirming that these rainfall-triggered
landslides are generally preceded by an increase in total precipitation depth.
Notably, when considering all mass movements together
(Fig. \ref{fig:percentile}) the precipitation events that triggered
landslides at burned sites were significantly smaller than those that
triggered mass movements at unburned locations (Mann--Whitney test, \(95\,\%\)
confidence). At first glance, this difference supports the overarching 
hypothesis that wildfire does in
fact increase mass movement susceptibility, since mass movements in the period
after a fire can be triggered by less precipitation than might normally
be required to cause mass movement. However, an examination of each region
separately reveals that the difference in precipitation percentile
between burned and unburned sites is present in some regions but not in
others (see Fig. \ref{fig:percentile}). The California area
(Fig. \ref{fig:percentile} panel (b)) has a particularly strong signal,
whereas tropical regions do not show any significant decrease between
precipitation at burned and unburned sites or display the reverse effect
of higher precipitation percentiles for unburned locations than burned
locations. In total, these initial results suggest that post-wildfire 
landslides are isolated to areas, such as California, where such 
cascading hazards have been repeatedly observed.


\begin{figure}
    \centering
    \includegraphics[width=.5\textwidth]{figures/pvalues_location_accuracy.png}
    \caption{p-values for Mann-Whitney hypothesis tests comparing precipitation percentiles at burned and unburned sites. The thick black line shows the p-values for all mass movements, while green and orange lines show high (1 km or less) and low (greater than 1 km) location accuracies. A horizontal black line shows the p=0.05 significance threshold, while a vertical black line indicates the day of the event.}
    \label{fig:p-location}
\end{figure}

Figure \ref{fig:p-location} shows p-values for Mann-Whitney hypothesis tests comparing precipitation percentiles for burned and unburned groups for high and low location accuracy groups of mass movements. High accuracy indicates less than 1 km. Several regions, such as California (Fig. \ref{fig:p-location} panel (b)) show substantial differences between the high-accuracy and low-accuracy p-values. Sample sizes of burned locations among the exact locations are low, ranging from 2 to 34 in each region, with overall only 3.7\% of high-accuracy  mass movements classified as burned (below the threshold used to exclude regions from this study). The low percentage of burned sites may partially account for high p-values among the high-accuracy group. An additional important consideration is the likelihood of a greater number of false positive burned sites among the low-accuracy group. Notably, the percentage of identified burned sites using this method increases with the location accuracy radius – globally 12.5\% of low-accuracy  mass movements were identified as burned in contrast with only 3.7\% of high-accuracy  mass movements.

\begin{figure}
    \centering
    \includegraphics[width=.5\textwidth]{figures/pvalues_debrisflow.png}
    \caption{p-values of Mann-Whitney tests comparing  mass movement-triggering precipitation percentiles at burned and unburned sites. The black line shows results for all  mass movements, while debris flows and other  mass movements are shown in green and orange respectively. A horizontal black line shows a 95\% confidence level for the hypothesis test, and a vertical black line indicates the day of the  mass movements}
    \label{fig:p-debrisflow}
\end{figure}

Figure \ref{fig:p-debrisflow} shows the p-values of Mann-Whitney hypothesis tests, similarly to those performed for Fig. \ref{fig:percentile}. The results in Fig. \ref{fig:p-debrisflow} are split into categories by  mass movement type, with ‘debris flow’ and ‘mudslide’ types labeled as debris flows and all other types labeled as other. With the exception of the Pacific Northwest (Fig. \ref{fig:p-debrisflow} panel (d)), the mass movement type has limited impact on the number of days with significant differences ($p < 0.05$) in precipitation in the 14 days prior to the mass movement in regions with any such significant differences. For example, in California (Fig. \ref{fig:p-debrisflow} panel (b)), nine days have a statistically significant difference for both groups. In the Intermountain West eight days have a statistically significant difference for debris flows while similarly six days have a statistically significant difference for other types of mass movements.

\subsection{Comparison of bootstrapped samples and pre-landslide precipitation}
\label{sec:result-bootstrap}

\begin{figure}
\centering
\includegraphics[width=.9\textwidth]{figures/precip_bootstrap_wilcox+kde.png}
\caption{p-values of Mann--Whitney hypothesis tests comparing mass movement-triggering  precipitation relative to 100 bootstrapped samples (n~100 for each sample) drawn from a 38-year precipitation record from the mass movement locations. The y-axes are shown with a probit transform to expand the section of the axis where p-values are below 0.05 (significant at 95\% confidence, shown as a dashed black line). The y-axis has also been inverted so that larger differences in precipitation (lower p-values) are higher on the y-axis for consistency with the percentile plots in Fig. \ref{fig:percentile}. In panels (h)-(u), an example of the kernel density estimate (kde) for day-of-landslide precipitation in black separated by burned and unburned groups is compared with kdes of all bootstrapped samples in orange (burned group) or purple (unburned group).}
\label{fig:bootstrap}
\end{figure}

Figure \ref{fig:bootstrap} highlights the increase in precipitation in the
days before a mass movement relative to historical amounts for that location
and time of year, i.e., relative to climatology, offering a robust
assessment of the mass movement precipitation departure. The Mann--Whitney 
p--values comparing the precipitation record on each day to each of the \(\sim100\) 
samples are shown in \ref{fig:bootstrap} panels (a)--(g). Mass movement events have 
been split into burned and unburned groups (shown in orange and purple respectively) 
for six regions and for all mass movements in the study. Bootstrapped samples were 
drawn from the same DOY and locations as the mass movements but from a randomly 
selected year. In panels (a)-(g), box plots of p–values represent the degree to 
which the mass movement-triggering precipitation differed from climatological 
precipitation with lower p-values indicating a more significant difference between 
the two precipitation distributions. The Mann-Whitney tests were directional, so 
only differences where the precipitation is greater than would be expected result 
in low p-values. Examples of the kernel 
density estimates of each bootstrap sample as compared to the precipitation on the 
day of the mass movement are shown in Fig. \ref{fig:bootstrap} panels (h)--(u) to better 
illustrate the comparisons made by the hypothesis tests  in panels (a)--(g). Each 
orange or purple curve was tested against the black curve to obtain the boxplots of 
p--values at 0 days before the mass movement. A clear difference
between burned and unburned sites is shown for the same regions as in
Fig. \ref{fig:percentile}, but with the addition of Southeast Asia.
Beyond the emergence of a signal in Southeast Asia, additional
differences between regions in the timing of precipitation in the period
leading up to the mass movement are visible in Fig. \ref{fig:bootstrap} panels (a)--(g) that
were not clear in the precipitation percentile analysis.

Different storm timing is apparent among the regions, and between the burned and 
unburned sites of the same region. Firstly, in California and Southeast Asia 
(Fig. \ref{fig:bootstrap} panels (b) and (g))
)), we see a similar pattern where relative precipitation at unburned sites is consistently higher than at burned sites. Nonetheless, for both burned and unburned sites, the rise in precipitation takes place over a similar amount of time (approximately 5 days). Curiously, unlike in California, the bootstrap analysis reveals a long-term difference between burned sites and unburned sites in the Mann–Whitney p–value for Southeast Asia despite location-specific normalization, suggesting that the mass movements at unburned locations might be primarily triggered in years that are wetter than usual on a monthly or seasonal scale. In the Pacific Northwest 
(Fig. \ref{fig:bootstrap} panel (d)), 
the precipitation at the burned sites does not become significantly larger than climatology until the day of the mass movement. The Mann–Whitney p–values for the burned group remain well above 0.05 just days before the mass movement as the p–value for the unburned group begins to fall. Under the assumption that shorter storms are associated with runoff-driven mass movements while longer storms that allow more time for the soil column to saturate are associated with infiltration-driven mass movements, this difference in storm timing could reflect that in the Pacific Northwest the burned mass movement locations are largely runoff-driven while mass movements at unburned locations are infiltration-driven 
\citep{cannonWildfirerelatedDebrisFlow2005}.
The Mann--Whitney p--values for the burned group remain well above \(0.05\) just days before the mass movement as the p--value for the
unburned group begins to fall. In the Intermountain West
(Fig. \ref{fig:bootstrap} panel (c))
antecedent precipitation for the
burned group is generally characterized by a dry spell going back thirty
days or more. In this region, thirty to twenty days before the mass movement p--values for burned sites are
consistently above \(0.9\), suggesting a high likelihood (\(>90\,\%\)) that there
was less precipitation than usual during that time. During the same period, the p-values at unburned sites remain close to 0.05


\subsection{Landslide and fire seasonality experiment}
\label{sec:result-seasonality}

\begin{figure}
\centering
\includegraphics[width=0.7\textwidth]{figures/fire_to_landslide.png}
\caption{DOY of mass movements, DOY of fires, and the length of time in between fire and mass movement by region. Each horizontal line represents one event, arranged on the y-axis in order of the length of the delay between wildfire and mass movement. Black dots on the right show the day of the year the mass movement occurred, and horizontal lines represent the duration of time elapsed in between the fire and the mass movement. Lines are colored by the season of the fire and are ordered by the day of the fire relative to the mass movement. The black lines, or rug, at the top of each panel as well as the colored rug on the left duplicate the day-of-year of the fires to highlight seasonal patterns.}
\label{fig:fire-landslide-timing}
\end{figure}

Figure \ref{fig:fire-landslide-timing} shows the seasonality of fires and mass movements
at each site, in addition to the length of time elapsing between the
fire and the mass movement. Landslides in several regions, especially
California and the Himalayas, tend to occur at the same time of year.
This time of year, for the regions where it exists, will be referred to
as `landslide season.' Similarly, nearly all of the regions have a fire
season, which is most clearly visible in the black rug at the top of
each panel in Fig. \ref{fig:fire-landslide-timing}. Figure 
\ref{fig:fire-landslide-timing} panel (a) shows that 
fires occur nearly year-round when considering all regions together, 
but the other panels in Fig. \ref{fig:fire-landslide-timing} show that within any particular region, 
fires occur only during a distinct time of year. However, the delay between 
fire and mass movement is not consistently equal to the length of time between 
fire season and the following mass movement season. The mass movements are distributed 
such that \(53\,\%\) occur
within a one year after the fire. Since both mass movements and fires have seasonal patterns, the
typical delay between fire and mass movement for each region appears to be
primarily related to the relationship between fire season and mass movement
season. For example, California has a long fire season and a shorter
landslide season, and so when fires occur at the end of winter,
immediately after mass movement season, there is typically a longer delay
before the mass movement than when fires occur immediately before mass movement
season. By contrast, in the Himalayas the delay between fire and
landslide is relatively uniform due to a shorter fire season that does
not overlap with the mass movement season. In general, the mass movements occur during the period of greatest rainfall, such as the winter in California and the summer in the Himalayas (see Figure \ref{fig:map} for regional precipitation climatology). The seasonal pattern of post-wildfire mass movements is to some degree determined by an interaction between fire seasonality and precipitation seasonality. 

\begin{figure}
    \centering
    \includegraphics[width=.5\textwidth]{figures/pvalues_timing.png}
    \caption{p-values for Mann-Whitney hypothesis tests comparing precipitation percentiles at burned and unburned sites. The thick black line shows the p-values for all mass movements, while orange and green lines show mass movements occurring within one year of a wildfire and between one and three year of a wildfire respectively. A horizontal black line shows the $p=0.05$ significance threshold, while a vertical black line indicates the day of the mass movement.}
    \label{fig:p-timing}
\end{figure}

Figure \ref{fig:p-timing} shows the p-values of Mann-Whitney tests comparing precipitation percentiles of groups of mass movements with different timing relative to wildfire with precipitation percentiles of mass movements at unburned sites. Landslides at burned sites were divided into two groups: within one year after a wildfire, mass movement between one and three years after a wildfire. In California and the Pacific Northwest of the US (Fig. \ref{fig:p-timing} panels (b) and (d)), the p-values are similar among the two timing groups. By contrast, in the Intermountain West of the US (Fig. \ref{fig:p-timing} panel (c)), the lower precipitation percentiles at burned sites are only statistically significant at the time of the for mass movements occurring 1-3 years after a wildfire. However, precipitation is significantly lower in the ‘less than one year’ group in the seven-to-three days before the mass movement. In Central America, the Himalayas, and Southeast Asia (Fig. \ref{fig:p-timing} panels (e), (f), and (g)), differences between burned and unburned sites are not statistically significant for either group.

\begin{figure}
\centering
\includegraphics[width=0.9\textwidth]{figures/precip_frequency_seasonality.png}
\caption{Precipitation frequency anomaly relative to the long-term mean aligned by the mass movement date. In panels (a)(g), frequency is shown both daily and smoothed with a 90-day moving average to highlight shifts. Daily precipitation frequency is represented as thin lines in orange and purple (burned and unburned groups) while the 90-day average is a thicker line. The long-term mean has been removed from all the frequency curves. Landslides are in burned and unburned groups for each region separately and for all mass movements. In panels (h)--(n), the kernel density estimate of mass movements by the time of year is shown for both the burned and unburned groups in a radial plot.}
\label{fig:seasonality}
\end{figure}

Figure \ref{fig:seasonality} shows differences in seasonality
between burned and unburned mass movement seasonality on the right and the
results of the precipitation frequency analysis on the left. The kernel density estimates on the right show changes in the seasons (e.g. Fall or Winter) in which landslides at burned and unburned sites occurred. By contrast, the analysis on the left shows when landslides in each group tended to occur relative to the times of year with greater precipitation frequency. While all
regions except for Central America (Fig. \ref{fig:seasonality}
panel (l)) display some kind of shift in seasonality between burned and
unburned mass movements in right-hand panels of Fig. \ref{fig:seasonality} ((h)--(n)), the magnitudes
and directions of these shifts varies by region. Interestingly, the
regions with clear shifts in seasonality have shifts of different
directions, i.e. earlier or later in the year, and magnitudes, i.e. a few
weeks to half a year. In the Southeast Asia
(Fig. \ref{fig:seasonality} panel (n)), mass movements at burned
sites happen in the summer rather than the winter for unburned sites, a
6-month shift. In contrast, mass movements in the Intermountain West
(Fig. \ref{fig:seasonality} panel (j)), burned mass movements appear
to happen in the spring while unburned mass movements occur in the winter, a
3-month shift later in the year. In California
(Fig. \ref{fig:seasonality} panel (i)), by contrast, burned
landslides are shifted earlier in the year and by only a few weeks, with
both burned and unburned mass movements occurring primarily in the fall and
early winter. Finally, in the Pacific Northwest
(Fig. \ref{fig:seasonality} panel (k)), it appears that some of
the burned mass movements occur in the usual mass movement season of fall and
early winter, while another peak lies 6 months away at the beginning of
summer.

The precipitation frequency in Fig. \ref{fig:seasonality}
panels (a)--(g) highlights differences in when mass movements tend to occur
relative to the wetter parts of the annual precipitation cycle between
burned and unburned groups. A curve for burned sites that is shifted
slightly to the right of the corresponding curve for unburned sites, as
is the case for the burned group precipitation frequency in the
California region (Fig. \ref{fig:seasonality} panel (b)),
indicates that burned landslides occurred earlier in the rainy
season. In California 
(Fig. \ref{fig:seasonality} panel (b)) burned mass movements are clearly
shifted to a period earlier in the year with more frequent precipitation,
i.e. earlier in the wet season, although the shift is larger in
California. This provides evidence confirming our hypothesis that wildfire increases mass movement susceptibility in these regions, since it suggests that a smaller precipitation trigger that might be found earlier in a wetter part of the year is required to trigger a mass movement after a fire. The Intermountain West
(Fig. \ref{fig:seasonality}, panel (c)) also has a pronounced
seasonal shift, but in this case the shift is much larger, so much so
that the burned mass movements in this region appear to occur as a result of
a large storm in the middle of a dry part of the year. Other regions
(Pacific Northwest in panel (d), Southeast Asia in panel (g), and Central
America in panel (e)) show differences in the magnitude of the annual
cycle in precipitation frequency, but no shift in seasonality. These
magnitude changes are not consistent in direction or degree across
regions. In Southeast Asia, where Fig. \ref{fig:seasonality}
panel (n) shows a shift in seasonality but panel (g) does not show a shift
relative to the wetter parts of the year, these results suggest that
there could be a spatial or climatic bias to the locations of burned
landslides that is causing the seasonal difference.


\section{Discussion}\label{discussion}

The results of this study suggest that while post-wildfire mass movements are associated with shifts in the magnitude, timing, and seasonality of storms relative to other mass movements, these effects are not consistent across regions.  Globally, there
are clear differences in the percentiles of mass movement-triggering storms
(see Fig. \ref{fig:percentile}), with mass movements in burned areas
often triggered by comparatively smaller storms. At first glance, this supports the hypothesis that fires increase mass movement susceptibility, since a smaller precipitation trigger is sufficient to cause a mass movement. However, this trend is largely driven by the California region and to a lesser extent the Intermountain West and Pacific Northwest of North America. In Central America/Caribbean, Southeast Asia, and the Himalayan regions there is an increase in rainfall relative to climatology leading up to the mass movement, but there is no significant difference between relative precipitation depths based on fire history. 
The original percentile analysis includes only wet days; the bootstrap analysis takes into account both wet and dry days.

Differences in the mass movement-triggering storms relative to their precipitation climatology shown by the bootstrap analysis 
(Fig. \ref{fig:bootstrap}) confirm the results of the original percentile analysis, with two notable exceptions. In Southeast Asia, the bootstrap analysis indicates that the burned sites had smaller precipitation triggers relative to climatology despite no significant difference in the first analysis between the wet-day precipitation percentiles. This discrepancy suggests that there may be a precipitation frequency bias between burned and unburned sites in this region. In addition, burned locations in the Pacific Northwest
appear to be associated with rainfall that began closer to when the
landslide occurred (Fig. \ref{fig:bootstrap} panel (d)). This raises the possibility that mass movements at burned sites in this region are caused more often than in unburned locations by runoff instead of infiltration. More information is needed, for example about the antecedent soil moisture at these locations. This result
is consistent with previous research suggesting that post-wildfire debris
flows are predominantly triggered by runoff-driven erosion as a result
of shorter and more intense storms in the Western US 
\citep[\(76\,\%\)][]{cannonWildfirerelatedDebrisFlow2005}; however in that case we would have expected to see a similar pattern in the California and Intermountain West regions. 

Many of the mass movements at burned locations in the Intermountain West (Fig. \ref{fig:bootstrap} panel (c)) appear to be
particularly susceptible to shorter-duration storms that occur after a
dry spell stretching from thirty to twenty days before the mass movement and
possibly even further back in time. A similar pattern of low frequency precipitation
followed by a sharp spike can be seen in the burned locations in
Fig. \ref{fig:seasonality} panel (c). One possible explanation
is that dry, recently burned soil is particularly erosive in those
areas. An example of drought conditions contributing to a landslide is described by \citep{handwerger2019shift}. These differences are also due in part to the different regional
climates, with the California and Pacific Northwest regions having more
clearly defined longer-duration rainy seasons, relative to the more
variable and sporadic precipitation seasonality of the Intermountain
West.

Different combinations of fire season, mass movement season, and any
overlap between the two may be an important driving factor in the degree
to which fires increase mass movement susceptibility. For example, in places
where the wet season begins towards the end or immediately after fire
season, such as the Intermountain West, California, and the Himalayas,
the landscape has no time to recover from the fire before mass movement
season begins and therefore burned locations may be much more
susceptible (see Fig. \ref{fig:fire-landslide-timing} panels (b), (c), 
and (f)). On the other hand, in regions like the Pacific Northwest,
Central America, and Southeast Asia (Fig. \ref{fig:fire-landslide-timing} 
panels (d), (e), and (g)), where mass movement season is not as
well defined, it is more likely that the landscape could at least
partially recover before a triggering storm occurs.

Some of the regions that did not display a significant difference in
percentile nonetheless showed a shift in the timing of burned mass movements
relative to their respective annual pattern of precipitation (see
Fig. \ref{fig:seasonality} panels (h)--(n)). The various types of shifts in
landslide seasonality are likely reflective of the different effect of
fires. A shift of the mass movement season to slightly earlier in the year,
such as was noticeable in California and the Himalayas (see
Fig. \ref{fig:seasonality} panels (i) and (m)) supports the hypothesis that
wildfire increases mass movement susceptibility because it suggests that fewer or 
smaller precipitation events earlier in the season are sufficient to trigger a 
mass movement. The Intermountain West
(Fig. \ref{fig:seasonality} panel (j)) also has a pronounced seasonal
shift, but in this case the shift is much larger and in the opposite
direction: burned mass movements appear to occur an entire season later than
unburned mass movement, falling in the driest part of the year instead of
the wettest. This corresponds to the evidence from the bootstrap
analysis suggesting that dried out soil or slow vegetation regrowth may
be an important part of the post-wildfire mass movement mechanism in this
region. Vegetation regrowth as a main control of mass movement susceptibility
is supported by a study of mass movement occurrence in the San Gabriel
mountains of the US by \citet{rengersLandslidesWildfireInitiation2020}, in which the authors found that hillslopes
with slower vegetation regrowth were more likely to have mass movements.

A similar trend to the Intermountain West in terms of seasonal shift is
visible for some, but not all, of the mass movements in the Pacific
Northwest (Fig. \ref{fig:seasonality} panel (k)), suggesting perhaps that
some of mass movements in that region would have been better categorized as
part of the Intermountain West region. In Southeast Asia
(Fig. \ref{fig:seasonality} panel (n)) there also appears to be a seasonal
shift similar to that of the Intermountain West, but it is not matched
by a shift relative to the annual precipitation frequency pattern
(Fig. \ref{fig:seasonality} panel (g)). This suggests that the seasonality
``shift'' in Southeast Asia due to spatial bias in fire occurrence. Further study of variation in climate across this region is needed. Finally, Central
America (Fig. \ref{fig:seasonality} panel (l) has very similar precipitation
frequency in burned and unburned locations. Since there is little
difference between the precipitation frequency or magnitude (see
Figs. \ref{fig:percentile} panel (e), \ref{fig:seasonality} panel (e)) in this area,
it is possible that there are many misidentified false positive
post-wildfire mass movements in Central America, perhaps due to the generally low location accuracy in that region. It is also possible
wildfire does not have as much of an effect on mass movement susceptibility
in that region.

The timing of mass movements relative to wildfire may also influence the magnitude of triggering storms. While in some regions, such as California and the Pacific Northwest, timing does not have a major impact on precipitation percentile differences, the Intermountain West of the US displays two distinct behaviors depending on the timing of mass movements relative to wildfire. In the year immediately after a fire, the precipitation percentile is lower than for mass movements at unburned locations in the seven-to-three days before the mass movement, before rising to match precipitation percentile at unburned locations (see Figure \ref{fig:p-timing} panel (c)). This pattern matches the result from Figure \ref{fig:seasonality} panel (c) in which post-wildfire mass movements in this region appear to manifest as a large storm preceded by a period of infrequent precipitation. In contrast, timing appears to make little difference to the precipitation percentile in other regions.

Low mass movement location accuracy and lower number of burned mass movements
may have also contributed to the lack of conclusive results in the
Pacific Northwest, Southeast Asia and Central America. The regions
outside the US and Canada tended to have less accurate mass movement locations, and
less accurate locations were also more likely to be marked as burned.
Furthermore, less accurate locations were also more likely to be marked as burned, with a threefold increase in the percentage of mass movements identified as burned between high- and low-accuracy groups.
This is because larger mass movement radii were more likely to contain
burned area by chance alone, and hence become false positive
post-wildfire mass movements, i.e.~landslides that occurred nearby but not
coincident to a burned area. This idea is supported by the lower
cumulative burned fractions within the regions outside the US and Canada (see
Fig. \ref{fig:map} panels (c) and (d)). Though mass movement accuracy in the GLC is an approximate measure, introducing the possibility of false negative unburned sites, false positive post-wildfire mass movements nonetheless represent a major potential source of error in this analysis. These uncertainties introduce the possibility that some of differences in triggering precipitation percentiles between burned and unburned sites may be related to unique qualities of fire-prone areas rather than fire itself.
Future studies
using visible and other satellite imagery to pinpoint mass movement
locations and dates could help clarify the post-wildfire posterior
landslide probability by essentially eliminating the location error.
Furthermore, there is a body of research that uses GIS data such as
slope or underlying lithography in combination with a statistical model
like a classification tree or logistic regression to assess mass movement
hazards \citep[e.g.][]{felicisimoMappingLandslideSusceptibility2013, leeApplicationVerificationFuzzy2007, ohlmacherUsingMultipleLogistic2003}, 
including some focused on post-wildfire mass movements 
\citep{cannonPredictingProbabilityVolume2010}. The introduction of 
such control datasets of confirmed unburned
landslide locations would also allow the use of additional variables
like slope, land use, and aridity index to be incorporated into a model
as part of an assessment of which properties of sites have the greatest
influence on changes in mass movement susceptibility at burned sites.


\conclusions

Clear differences were shown between rainfall-triggered mass movements at unburned and unburned locations in the magnitude of precipitation triggers, the seasonality of mass movements, and the timing of triggering storms. These findings suggest that wildfires increase susceptibility to mass movements, especially in regions of the Western US. However, they also suggest that post-wildfire mass movements are not a spatially uniform phenomenon. Both the mechanisms by which burned mass movements are triggered and the degree to which wildfire increases susceptibility varies by region. 

The precipitation percentile immediately before a mass movement was found to
be smaller at burned locations for all regions combined, as well as for
the California, Intermountain West, and Pacific Northwest regions, but
not for the others. This result suggests greater mass movement
susceptibility in those three regions following a wildfire. In
California and the Pacific Northwest, mass movement-triggering storms tended
to be shorter at burned locations, suggesting that these mass movements are
more often runoff-driven than mass movements at unburned locations. In
contrast, in the Intermountain West burned mass movement locations appear to be
characterized by a dry spell of at least 20 days followed by a sharp uptick in
precipitation, suggesting that burned and dry soil may be the most
vulnerable to extreme erosion in that region. Finally, shifts in
landslide seasonality were noted in every region except Central America,
although the characteristics of these shifts were not consistent among
regions. In some regions such as California and the Himalayas,
landslides at burned locations occurred earlier in the wet season,
suggesting greater susceptibility to mass movements caused by fire. In other
regions such as the Intermountain West and Southeast Asia, mass movement
seasonality was shifted by 3 or 6 months, suggesting that the conditions resulting in mass movements differ in more fundamental ways at burned sites. For example, in the Intermountain West we posit that a portion of post-wildfire
mass movements may be caused by isolated intense
thunderstorms on dry soil producing the observed pattern of mass movement-triggering storms in burned locations preceded by at least several weeks with limited precipitation. Among the unburned sites, by contrast, a pattern of mass movements occurring during the wettest part of the year suggests that saturation of the soil is a more important precursor.

Developing a better understanding of the ways in which mass movement hazards vary around the world is important for mitigation efforts as well as predicting how mass movement hazards will respond to a changing climate. Data acquisition is a major barrier to this type of global analysis of mass movement statistics. Both precipitation and burn status are major sources of uncertainty in this analysis due to imprecise mass movement locations. This work offers new insights into the role of wildfire on mass movement susceptibility, representing a first step towards broader understanding of regional triggering mechanisms. Future efforts should incorporate additional high-accuracy mass movement locations (e.g. \(\sim500\text{m}\)) that are more representatively distributed around the globe to further advance understanding into mass movement responses across climates and regions. 


\authorcontribution{Elsa Culler and Ben Livneh designed the experiments in consultation with all co-authors. Balaji Rajagopalan assisted with the design of the statistical analysis and Kristy Tiampo aided in the analysis of the mass movement triggers. Elsa Culler processed the data, developed the  model code and performed the statistical analysis. Elsa Culler prepared the manuscript with contributions from all co-authors.} %% this section is mandatory

\competinginterests{The authors declare that they have no conflict of interest.} %% this section is mandatory even if you declare that no competing interests are present

\begin{acknowledgements}
This research was funded by NASA IDS grant 16-IDS16-0075, The Interaction of Mass Movements with Natural Hazards Under Changing Hydrologic Conditions.
\end{acknowledgements}

%% REFERENCES
\bibliographystyle{copernicus}
\bibliography{regions}

%% Since the Copernicus LaTeX package includes the BibTeX style file copernicus.bst,
%% authors experienced with BibTeX only have to include the following two lines:
%%
%%
%% URLs and DOIs can be entered in your BibTeX file as:
%%
%% URL = {http://www.xyz.org/~jones/idx_g.htm}
%% DOI = {10.5194/xyz}


%% LITERATURE CITATIONS
%%
%% command                        & example result
%% \citet{jones90}|               & Jones et al. (1990)
%% \citep{jones90}|               & (Jones et al., 1990)
%% \citep{jones90,jones93}|       & (Jones et al., 1990, 1993)
%% \citep[p.~32]{jones90}|        & (Jones et al., 1990, p.~32)
%% \citep[e.g.,][]{jones90}|      & (e.g., Jones et al., 1990)
%% \citep[e.g.,][p.~32]{jones90}| & (e.g., Jones et al., 1990, p.~32)
%% \citeauthor{jones90}|          & Jones et al.
%% \citeyear{jones90}|            & 1990



%% FIGURES

%% When figures and tables are placed at the end of the MS (article in one-column style), please add \clearpage
%% between bibliography and first table and/or figure as well as between each table and/or figure.

% The figure files should be labelled correctly with Arabic numerals (e.g. fig01.jpg, fig02.png).


%% ONE-COLUMN FIGURES

%%f
%\begin{figure}[t]
%\includegraphics[width=8.3cm]{FILE NAME}
%\caption{TEXT}
%\end{figure}
%
%%% TWO-COLUMN FIGURES
%
%%f
%\begin{figure*}[t]
%\includegraphics[width=12cm]{FILE NAME}
%\caption{TEXT}
%\end{figure*}
%
%
%%% TABLES
%%%
%%% The different columns must be seperated with a & command and should
%%% end with \\ to identify the column brake.
%
%%% ONE-COLUMN TABLE
%
%%t
%\begin{table}[t]
%\caption{TEXT}
%\begin{tabular}{column = lcr}
%\tophline
%
%\middlehline
%
%\bottomhline
%\end{tabular}
%\belowtable{} % Table Footnotes
%\end{table}
%
%%% TWO-COLUMN TABLE
%
%%t
%\begin{table*}[t]
%\caption{TEXT}
%\begin{tabular}{column = lcr}
%\tophline
%
%\middlehline
%
%\bottomhline
%\end{tabular}
%\belowtable{} % Table Footnotes
%\end{table*}
%
%%% LANDSCAPE TABLE
%
%%t
%\begin{sidewaystable*}[t]
%\caption{TEXT}
%\begin{tabular}{column = lcr}
%\tophline
%
%\middlehline
%
%\bottomhline
%\end{tabular}
%\belowtable{} % Table Footnotes
%\end{sidewaystable*}
%
%
%%% MATHEMATICAL EXPRESSIONS
%
%%% All papers typeset by Copernicus Publications follow the math typesetting regulations
%%% given by the IUPAC Green Book (IUPAC: Quantities, Units and Symbols in Physical Chemistry,
%%% 2nd Edn., Blackwell Science, available at: http://old.iupac.org/publications/books/gbook/green_book_2ed.pdf, 1993).
%%%
%%% Physical quantities/variables are typeset in italic font (t for time, T for Temperature)
%%% Indices which are not defined are typeset in italic font (x, y, z, a, b, c)
%%% Items/objects which are defined are typeset in roman font (Car A, Car B)
%%% Descriptions/specifications which are defined by itself are typeset in roman font (abs, rel, ref, tot, net, ice)
%%% Abbreviations from 2 letters are typeset in roman font (RH, LAI)
%%% Vectors are identified in bold italic font using \vec{x}
%%% Matrices are identified in bold roman font
%%% Multiplication signs are typeset using the LaTeX commands \times (for vector products, grids, and exponential notations) or \cdot
%%% The character * should not be applied as mutliplication sign
%
%
%%% EQUATIONS
%
%%% Single-row equation
%
%\begin{equation}
%
%\end{equation}
%
%%% Multiline equation
%
%\begin{align}
%& 3 + 5 = 8\\
%& 3 + 5 = 8\\
%& 3 + 5 = 8
%\end{align}
%
%
%%% MATRICES
%
%\begin{matrix}
%x & y & z\\
%x & y & z\\
%x & y & z\\
%\end{matrix}
%
%
%%% ALGORITHM
%
%\begin{algorithm}
%\caption{...}
%\label{a1}
%\begin{algorithmic}
%...
%\end{algorithmic}
%\end{algorithm}
%
%
%%% CHEMICAL FORMULAS AND REACTIONS
%
%%% For formulas embedded in the text, please use \chem{}
%
%%% The reaction environment creates labels including the letter R, i.e. (R1), (R2), etc.
%
%\begin{reaction}
%%% \rightarrow should be used for normal (one-way) chemical reactions
%%% \rightleftharpoons should be used for equilibria
%%% \leftrightarrow should be used for resonance structures
%\end{reaction}
%
%
%%% PHYSICAL UNITS
%%%
%%% Please use \unit{} and apply the exponential notation


\end{document}
